\documentclass{article}

% begin MULTILINGUAL SUPPORT
% many thanks to https://tex.stackexchange.com/users/7537/ at https://tex.stackexchange.com/questions/5076/is-it-possible-to-keep-my-translation-together-with-original-text

% enables babels language support
\usepackage[ngerman, english]{babel}

% import sty file
\usepackage{multilang}

% set short and long language codes, the second argument must be known by babel
\setdoclang{en}{english}
%\setdoclang{de}{ngerman}

% end MULTILINGUAL SUPPORT

% set paper size, orientation and margins
\usepackage[a4paper, portrait, margin=1.5cm]{geometry}

% empty lines between paragraphs instead of first line indentation
\usepackage{parskip}

% justify text
\usepackage{ragged2e}
\justifying

% draw flow charts
\usepackage{tikz}
\usetikzlibrary{shapes.geometric, arrows}

% clickable links, links with text and \nameref
\usepackage{hyperref}

% use \url tag
\usepackage{url}

% use preamble of main file in every sub file
% best loaded last in the preamble
\usepackage{subfiles}

% begin FLOW CHART BLOCKS
\tikzstyle{startstop} = [rectangle, rounded corners,
minimum width=3cm, minimum height=1cm,
text centered, text width=3cm,
draw=black, fill=red!30]

\tikzstyle{io} = [trapezium,
trapezium stretches=true,
trapezium left angle=70,
trapezium right angle=110,
minimum width=3cm, minimum height=1cm,
text centered, text width=3cm,
draw=black, fill=blue!30]

\tikzstyle{process} = [rectangle,
minimum width=3cm, minimum height=1cm,
text centered, text width=3cm,
draw=black, fill=orange!30]

\tikzstyle{decision} = [diamond,
aspect = 2,
minimum width=3cm, minimum height=1cm,
text centered, text width=2.5cm,
draw=black, fill=green!30]
\tikzstyle{arrow} = [thick,->,>=stealth]
% end FLOW CHART BLOCKS

% begin TITLE INFORMATION
\title{
    \lang{en}{
        Preparing ('Digitizing') Images with Inkscape and Ink/Stitch\\
        for Machine Embroidery
    }
    \lang{de}{
        Bilder mit Inkscape und Ink/Stitch\\
        für das Maschinensticken vorbereiten ('Digitalisieren')
    }
}
\author{Harald Kraft}
\date{
    \today
    \bigbreak
    % add license information to title
    \lang{en}{
        This entire document is licensed under the GNU Affero General Public License Version 3.\\
        \url{https://gnu.org/licenses/agpl.html}
    }
    \lang{de}{
        Dieses Dokument ist in Gänze unter der GNU Affero General Public License Version 3 lizensiert.\\
        \url{https://gnu.org/licenses/agpl.html}
    }
}
% end TITLE INFORMATION

\begin{document}

    \maketitle

    \tableofcontents

    \pagebreak

    \subfile{sections/1/prereq.tex}

        \subfile{sections/1/inkscape.tex}

        \subfile{sections/1/inkstitch.tex}

        \subfile{sections/1/usage.tex}

        \subfile{sections/1/setup.tex}

        \subfile{sections/1/ressources.tex}

    \pagebreak

    \subfile{sections/2/workflow.tex}

        \subfile{sections/2/overview.tex}

        \pagebreak

        \subfile{sections/2/import.tex}

        \subfile{sections/2/fileTypes.tex}

            \subfile{sections/2/vector.tex}

            \subfile{sections/2/raster.tex}

        \subfile{sections/2/resiize.tex}

        \subfile{sections/2/convert.tex} \label{vectorConversion}

        \subsection{Delete Background}
        %TODO

            \subsubsection{Delete Background Layer}
            %TODO

            \subsubsection{Delete "Holes"}
            %TODO

        \subsection{Break Apart Image}
        %TODO

        \subsection{Finding and fixing Problems}
        %TODO

            \subsubsection{Troubleshooter}
            %TODO

            \subsubsection{Clean up Document}
            %TODO

            \subsubsection{Small Infill}
            %TODO

            \subsubsection{Border Crosses Itself}
            %TODO

            \subsubsection{Rung mIntersects Rail more than Once (Satin Stitch)}
            %TODO

        \subsection{Simplify Paths}
        %TODO

        \subsection{Finding and Fixing Problems, Again}
        %TODO

        \subsection{Creating Outlines}
        %TODO

        \subsection{Adding Writing / Lettering}
        %TODO

        \subsection{Checking what the Stitched Image will look like}
        %TODO

        \subsection{Assigning Stitch Types}
        %TODO

            \subsubsection{Straight Stitch}
            %TODO

            \subsubsection{Satin Stitch}
            %TODO

            \subsubsection{Fill Stitch}
            %TODO

        \subsection{Reorganizing the Stitch Order}
        %TODO

            \subsubsection{Grouping same Colour Parts}
            %TODO

            \subsubsection{Avoiding visible Jump Stitches}
            %TODO

        \subsection{One last Check}
        %TODO

        \subsection{Exporting to a File your Embroidery Machine can understand}
        %TODO

\end{document}
